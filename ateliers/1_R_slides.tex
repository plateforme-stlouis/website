\documentclass[10pt, xcolor=table]{beamer}

\usepackage[utf8]{inputenc}
\usepackage[T1]{fontenc}
\usepackage[french]{babel}

\usepackage{amsmath}
\usepackage{color}
\usepackage{amssymb}
\usepackage{amsfonts}
\usepackage{pgfpages}

%% %% truc de Berty
%% \usepackage{import}

%ajouts perso:
\usepackage{t1enc,ae}
\usepackage{multicol}
\usepackage{multirow}
\usepackage{graphicx}
\usepackage{subfigure}
%\usepackage{caption}
%\usepackage{subcaption}
%\usepackage{hyperref}
%\usepackage{cancel}

% pour les symboles exotiques
%\usepackage{textcomp}
%\usepackage{pifont}

\usepackage{epstopdf}
%\usepackage{xcolor}
%\usepackage[table]{xcolor}
\definecolor{lightgray}{gray}{0.9}


\newcommand{\violet}{\textcolor{violet}}
\newcommand{\blue}{\textcolor{blue}}
\newcommand{\red}{\textcolor{red}}
\newcommand{\black}{\textcolor{black}}
\newcommand{\rose}{\textcolor{magenta}}
\newcommand{\cyan}{\textcolor{cyan}}
\newcommand{\green}{\textcolor{green}}

\pgfpagesuselayout{resize to}[a4paper,border shrink=5mm,landscape]

% theme classique
\mode<presentation>
%\useoutertheme{Madrid}
\usecolortheme{whale}
\usecolortheme{orchid}
\useinnertheme[shadow=true]{rounded}
\setbeamerfont{block title}{size={}}


% recouvrement automatique
\setbeamercovered{transparent}
% symboles de navigation vide (options: vertical,etc)
\setbeamertemplate{navigation symbols}{}

%
\setbeamerfont{institute}{size=\fontsize{9pt}{10pt}}

%% numero de page
\addtobeamertemplate{footline}{\hfill\insertframenumber/\inserttotalframenumber}

%% \AtBeginSection[]
%% {
%%   \begin{frame}[plain, noframenumbering]{Plan}
%%     \tableofcontents[currentsection]
%%  \end{frame}
%% }

%%%%%%%%%%%%%%%%%%%%%%%%%%%%%%%%%%
%%%%%%%%%%%%%%%%%%%%%%%



%%%%%
\title{%
  Introduction to R
}
\author{%
  %Simon Tournier
}
\date{November, 22 2018}

\institute{%
  \texttt{simon.tournier@univ-paris-diderot.fr}
}

\begin{document}

%%%%%%%%%%%%
\begin{frame}[plain, noframenumbering]
  \titlepage
\end{frame}
%%%%%%%%%%%%


\begin{frame}{What is R? -- analogy}

  \begin{tabular}{c|c}
    Analysis & Experiment
    \\
    \hline
    \begin{minipage}{0.49\textwidth}
      \begin{block}{Environment}
        \centering
        RStudio (text editor)
      \end{block}
      \begin{exampleblock}{Strategy}
        \centering
        Scripts, Plots and Commands
      \end{exampleblock}

      \begin{alertblock}{Raw}
        \centering
        Files \texttt{.R .r .Rmd}
      \end{alertblock}
    \end{minipage}
    &
    \begin{minipage}{0.49\textwidth}
      \begin{block}{Environment}
        \centering
        FlowJo / Diva
      \end{block}
      \begin{exampleblock}{Strategy}
        \centering
        Gatings and Plots
      \end{exampleblock}
      \begin{alertblock}{Raw}
        \centering
        Files \texttt{.fcs}
      \end{alertblock}
    \end{minipage}
  \end{tabular}
\end{frame}

\begin{frame}{About files}

  \begin{itemize}
  \item Files \texttt{.R .r .Rmd} are \blue{simple \textbf{text} files}
    (TextEdit,Notepad,Emacs\dots)
  \item Files \texttt{.fcs .wsp} are \emph{binary} files (as \texttt{.doc .xls} etc.)
  \end{itemize}
  \begin{alertblock}{}
    \begin{center}
      $\Longrightarrow$ you can \textbf{always open text} files, not binary ones.
    \end{center}
  \end{alertblock}

  \begin{itemize}
  \item Text files do not depend on a specific software.
  \item Binary files are only readable by specific softwares.
  \end{itemize}

  \begin{exampleblock}{}
    \begin{center}
      $\Longrightarrow$ What happens if now you have to pay them?
    \end{center}
  \end{exampleblock}

  \begin{block}{}
    \centering
    \textbf{Always open-able files means:}
    \begin{itemize}
    \item easy to share
    \item still access in the future\footnote{Try to open a Word document from
      1998. But I am reading text files from 1983.}
    \end{itemize}
  \end{block}
\end{frame}

\begin{frame}{What is R? II}

  \begin{alertblock}{Raw}
    \centering
    R is a scripting language
  \end{alertblock}
  \begin{exampleblock}{Strategy}
    \centering
    A script is a recipe to instruct how to process data
  \end{exampleblock}
  \begin{block}{Environment}
    \centering
    We write instructions inside an environment
  \end{block}

  \bigskip

  \begin{description}
  \item[Language] = syntax + grammar
  \item[Data] = how to represents an information
  \end{description}

  \begin{center}
    Code is law\footnote{Lawrence Lessig:
      \url{https://framablog.org/2010/05/22/code-is-law-lessig}}
    so \textbf{show me the details\footnote{The devil is in the details.}!}
  \end{center}

\end{frame}

%%%%%%%%%%%%%%
\end{document}
%%%%%%%%%%%%%%
%%%%%%%%%%%%%%
